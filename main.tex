\documentclass[a4paper,zihao=-4,AutoFakeBold,AutoFakeSlant]{ctexart}
\usepackage{sjtu-bachelor-midterm}

% =====================================================
%            --- 这里开始编写你的信息 ---
% =====================================================

% ==========================================
%    --- 封面信息 (Cover Information) ---
%        用于自动生成中期检查报告的封面
% ==========================================

% 论文题目
% 提示:如果题目太长,可以使用 \\ 强制换行以保持视觉美观
\covertitle{你的很长很长很很长很长很长的\\ 很厉害的论文题目}

% 学生姓名
\name{你的名字}

% 学生学号
\studentid{512345678901}

% 专业
\major{你的专业}

% 指导教师
\supervisor{你的导师}

% 学院(系)
\school{你的学院}


% ======================================================
%                 --- 文档开始 ---
% ======================================================

\begin{document}

% 生成封面页
\makecover

% 生成填表说明页
\makeinstruction

% 初始化正文格式
\initbodyformat

% ======================================================
%            --- 这里开始编写你的报告正文 ---
% ======================================================


\section{课题进展情况:}

制作本中期检查报告 \LaTeX{} 模板的初衷,是为校友们提供一个比传统 Word 更专业的排版方案。

\subsection{模板使用指南 (Template Guide)}

本模板已针对上海交通大学中期检查报告的格式规范,深度定制了字体、间距、标题样式及参考文献样式。

\subsubsection{数学公式 (Mathematics)} \label{math_section}

\LaTeX{} 采用了经典的数学公式语法。行内公式使用 \$ 包裹,如 $a^2 + b^2 = c^2$。

对于复杂的行间公式,可以使用 equation 环境:
\[
\mathcal{F}(\omega) = \int_{-\infty}^{\infty} f(t) e^{-i \omega t} dt
\]

你也可以编写矩阵或多行对齐公式:
\[
A = \begin{pmatrix} 1 & 2 \\ 3 & 4 \end{pmatrix}, \quad \sum_{i=1}^n i = \frac{n(n+1)}{2}
\]

\subsubsection{图像处理与交叉引用 (Images \& References)}

插入图片建议使用 \texttt{figure} 环境,这样可以自动生成“图 1”这样的标签并支持自动编号。

\begin{figure}[htb]
    \centering
    % 请确保 figures 文件夹下有对应的图片文件
    \includegraphics[width=0.5\textwidth]{figures/A.png}
    \caption{示例图片} \label{fig_A}
\end{figure}

如 \ref{fig_A} 所示,图片会自动居中。在正文中使用标签和引用即可实现自动跳转。

\subsubsection{表格设计 (Tables)}

本模板对表格进行了局部优化,使其更符合中文学术排版的“三线表”或“全框表”风格。

\begin{table}[htb]
    \centering
    \caption{课题研究详细进度安排表} \label{tab_schedule}
    \renewcommand{\arraystretch}{1.5}
    \begin{tabular}{|c|c|c|}
        \hline
        \textbf{阶段} & \textbf{研究内容} & \textbf{预期成果} \\
        \hline
        第一阶段 & 国内外文献综述与需求分析 & 提交开题报告定稿 \\
        \hline
        第二阶段 & 核心算法设计与仿真验证 & 发表高质量学术论文 \\
        \hline
        第三阶段 & 实验数据收集与论文撰写 & 完成学位论文初稿 \\
        \hline
    \end{tabular}
\end{table}

通过引用 \ref{tab_schedule},读者可以清晰地了解研究进度。

\paragraph{三线表展示}

三线表通常用于展示实验结果或对比数据,其结构简洁明了。

\begin{table}[htb]
    \centering
    \caption{不同算法性能对比(三线表示例)} \label{tab_3line}
    \renewcommand{\arraystretch}{1.5}
    \begin{tabular}{cccc}
        \hline
        \textbf{方法} & \textbf{准确率 (\%)} & \textbf{召回率 (\%)} & \textbf{F1 分数} \\
        \hline
        Baseline & 85.2 & 82.1 & 83.6 \\
        Proposed & 92.4 & 90.8 & 91.6 \\
        State-of-art & 93.1 & 91.2 & 92.1 \\
        \hline
    \end{tabular}
\end{table}

通过引用 \ref{tab_3line} 可以看到,三线表取消了所有竖线,使数据阅读更加流畅。

\subsubsection{参考文献管理 (Bibliography)}

本模板按照要求集成了 \texttt{gb-7714-2015} 标准。你只需要在同级目录下准备一个 \texttt{ref.bib} 文件。

引用方式非常简单:

\begin{itemize}
    \item 单个引用:该方法的可行性已在文献 \cite{MAGNET_CAR} 中得到证明。
    \item 多个引用:目前主流观点支持该结论 \cite{MAGNET_CAR,WATER_WHEEL}。
\end{itemize}

参考文献列表会自动根据你的引用顺序生成在文末。

\subsubsection{章节标签与快速跳转}

你可以通过在标题后添加 \texttt{\textbackslash label\{label\_name\}} 来定义标签。

例如,本文档“数学公式”的章节定义为 
\texttt{\textbackslash subsubsection\{数学公式 (Mathematics)\} \textbackslash label\{math\_section\}}。
我们现在可以轻松地通过 \texttt{\textbackslash ref\{math\_section\}} 
跳回该部分 \ref{math_section}。

\subsubsection{常用排版技巧}

\begin{itemize}
    \item \textbf{加粗}:使用 \texttt{\textbackslash textbf\{加粗内容\}}。
    \item \textit{斜体}:使用 \texttt{\textbackslash textit\{斜体内容\}}。
    \item \underline{下划线}:使用 \texttt{\textbackslash underline\{内容\}}。
    \item \colorbox{yellow}{高亮}:使用 \texttt{\textbackslash colorbox\{yellow\}\{内容\}}。
    \item \textcolor{red}{彩色文字}:使用 \texttt{\textbackslash textcolor\{color\}\{内容\}}。
\end{itemize}

\paragraph{列表}

\subparagraph{无序列表 (Unordered Lists)}

无序列表可以自动处理多级嵌套:

\begin{itemize}
    \item 第一级项目
    \begin{itemize}
        \item 第二级嵌套项目
        \begin{itemize}
            \item 第三级嵌套项目
        \end{itemize}
    \end{itemize}
    \item 并列的一级项目
\end{itemize}

\subparagraph{有序列表 (Ordered Lists)}

有序列表会自动处理编号逻辑:

\begin{enumerate}
    \item 第一项任务:收集很多很多很多很多很多很多很多很多很多很多很多很多很多很多很多很多文献数据。
    \item 第二项任务:建立数学模型。
    \begin{enumerate}
        \item 子任务 A:参数标定。
        \item 子任务 B:灵敏度分析。
    \end{enumerate}
    \item 第三项任务:撰写开题报告。
\end{enumerate}

\subsection{这是第2部分}

哈哈哈哈哈哈哈哈哈哈哈哈哈哈哈哈哈哈哈哈哈哈哈哈哈哈哈哈哈哈哈哈哈哈哈哈哈哈哈哈哈哈哈哈哈哈哈哈哈哈哈,哈哈哈哈哈哈哈哈哈哈哈哈哈哈哈哈哈哈哈哈哈哈哈哈哈哈哈哈哈哈哈哈哈哈哈哈哈哈哈哈哈哈哈哈哈哈哈哈哈哈哈,哈哈哈哈哈哈哈哈哈哈哈哈哈哈哈哈哈哈哈哈哈哈哈哈哈哈哈哈哈哈哈哈哈哈哈哈哈哈哈哈哈哈哈哈哈哈哈哈哈哈哈。

\subsubsection{这是2.1部分}

哈哈哈哈哈哈哈哈哈哈哈哈哈哈哈哈哈哈哈哈哈哈哈哈哈哈哈哈哈哈哈哈哈哈哈哈哈哈哈哈哈哈哈哈哈哈哈哈哈哈哈,哈哈哈哈哈哈哈哈哈哈哈哈哈哈哈哈哈哈哈哈哈哈哈哈哈哈哈哈哈哈哈哈哈哈哈哈哈哈哈哈哈哈哈哈哈哈哈哈哈哈哈,哈哈哈哈哈哈哈哈哈哈哈哈哈哈哈哈哈哈哈哈哈哈哈哈哈哈哈哈哈哈哈哈哈哈哈哈哈哈哈哈哈哈哈哈哈哈哈哈哈哈哈。

\paragraph{这是一个段落}

哈哈哈哈哈哈哈哈哈哈哈哈哈哈哈哈哈哈哈哈哈哈哈哈哈哈哈哈哈哈哈哈哈哈哈哈哈哈哈哈哈哈哈哈哈哈哈哈哈哈哈,哈哈哈哈哈哈哈哈哈哈哈哈哈哈哈哈哈哈哈哈哈哈哈哈哈哈哈哈哈哈哈哈哈哈哈哈哈哈哈哈哈哈哈哈哈哈哈哈哈哈哈,哈哈哈哈哈哈哈哈哈哈哈哈哈哈哈哈哈哈哈哈哈哈哈哈哈哈哈哈哈哈哈哈哈哈哈哈哈哈哈哈哈哈哈哈哈哈哈哈哈哈哈。

这是一个普通的段落,哈哈哈哈哈哈哈哈哈哈哈哈哈哈哈哈哈哈哈哈哈哈哈哈哈哈哈哈哈哈哈哈哈哈哈哈哈哈哈哈哈哈哈哈哈哈哈哈哈哈哈,哈哈哈哈哈哈哈哈哈哈哈哈哈哈哈哈哈哈哈哈哈哈哈哈哈哈哈哈哈哈哈哈哈哈哈哈哈哈哈哈哈哈哈哈哈哈哈哈哈哈哈,哈哈哈哈哈哈哈哈哈哈哈哈哈哈哈哈哈哈哈哈哈哈哈哈哈哈哈哈哈哈哈哈哈哈哈哈哈哈哈哈哈哈哈哈哈哈哈哈哈哈哈。

\vspace{2em}
\noindent\textbf{参考文献 References:}
\vspace{-0.2cm}
\printbibliography[heading=none]


\section{课题研究已取得的阶段性成果:}

\subsection{发表的期刊}

已经发表了 Nature 和 Science 各 500000 篇

\subsection{发表的会议}

各大顶级会议各发了 100000 篇


\section{存在的问题及解决思路:}

做的非常好,根本就不存在问题。


\section{下一阶段的工作计划和研究内容:}

我计划再发 Nature 和 Science 各 500000 篇。


% ====================================================
%            --- 生成中期检查报告的意见汇总页 ---
% ====================================================
% 1. 如果只想生成空的意见汇总页面
%    把下面的部分全部删掉,只保留 \makeopinionpage{}{}{none} 即可
%
% 2. 如果想要将意见填好
%    - 第一个{}: 指导教师意见。
%    - 第二个{}: 学院(系)意见。
%    - 第三个{}: 审查结果,可选值为 agree, disagree 或 none。
% ====================================================
\makeopinionpage{
% 指导教师意见:

写的非常非常非常非常非常非常非常非常非常非常非常非常非常非常非常非常非常非常非常非常非常非常非常非常非常非常非常非常非常非常非常非常非常非常非常非常非常非常非常非常非常非常非常非常非常非常非常非常非常非常非常非常非常非常非常非常非常非常非常非常非常非常非常非常非常好。

同意该生通过中期检查。

}{
% 学院(系)意见:

做的非常非常非常非常非常非常非常非常非常非常非常非常非常非常非常非常非常非常非常非常非常非常非常非常非常非常非常非常非常非常非常非常非常非常非常非常非常非常非常非常非常非常非常非常非常非常非常非常非常非常非常非常非常非常非常非常非常非常非常非常非常不错。

同意该生通过中期检查。

}{agree} % 这里填 agree, disagree 或 none

\end{document}
